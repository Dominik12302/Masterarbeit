\begin{figure}[H]
	\begin{addmargin}{-2cm}
	\centering
		\includegraphics[angle=0,width=170mm]{bilder/bp525s/MM12_110402_result.png}
	\caption{Gutes Ergebnis des ScS-Wellen Splitting f�r ein Event vom 2. April 2011 aufgezeichnet von Station MM12. Der untersuchte Zeitbereich um die ScS-Welle wurde mit START und END markiert.}
	\label{fig:MM12.110402_result}
	\end{addmargin}
\end{figure}
\begin{figure}[H]
	\begin{addmargin}{-2cm}
	\centering
		\includegraphics[angle=0,width=170mm]{bilder/bp525s/MM12_110608_result.png}
	\caption{Gutes Ergebnis des ScS-Wellen Splitting f�r ein Event vom 8. Juni 2011 aufgezeichnet von Station MM12. Der untersuchte Zeitbereich um die ScS-Welle wurde mit START und END markiert.}
	\label{fig:MM12_110608_result}
	\end{addmargin}
\end{figure}
\begin{figure}[H]
	\begin{addmargin}{-2cm}
	\centering
		\includegraphics[angle=0,width=170mm]{bilder/bp525s/MM13_110608_result.png}
	\caption{Gutes Ergebnis des ScS-Wellen Splitting f�r ein Event vom 8. Juni 2011 aufgezeichnet von Station MM13. Der untersuchte Zeitbereich um die ScS-Welle wurde mit START und END markiert.}
	\label{fig:MM13_110608_result}
	\end{addmargin}
\end{figure}
\begin{figure}[H]
	\begin{addmargin}{-2cm}
	\centering
		\includegraphics[angle=0,width=170mm]{bilder/bp525s/MM15_110902_result.png}
	\caption{Gutes Ergebnis des ScS-Wellen Splitting f�r ein Event vom 2. September 2011 aufgezeichnet von Station MM15. Der untersuchte Zeitbereich um die ScS-Welle wurde mit START und END markiert.}
	\label{fig:MM15_110902_result}
	\end{addmargin}
\end{figure}